% Options: [nometapage]
\documentclass{ltxdoc}
\usepackage[T1]{fontenc}
\usepackage[utf8x]{inputenc}
\usepackage{multicol}
\usepackage[tt=false, sf=false, type1=true]{libertine}
\usepackage[varqu,scaled=.88]{zi4}
\usepackage[libertine]{newtxmath}
\usepackage[tableposition=top]{caption}
\usepackage{fancyvrb}
\usepackage{hypdoc}
\usepackage{enumitem}
\setlist%
{%
 topsep=0pt,%
 labelsep=6pt,
 noitemsep,%
 leftmargin=*
}
\setlist[description]{font=\normalfont\sffamily\bfseries}
\setlength\parskip{10pt}

\begin{document}

\title{\texttt{frbaart }\LaTeX\ class for the \textit{Policy Hub Papers} publications}
\author{Deimantas Galčius, Linas Stonys}
\maketitle

\abstract{This package provides a class for typesetting Policy Hub Papers publications of the
Federal Reserved Bank of Atlanta.}

\tableofcontents

\section{Introduction}

The \texttt{frbaart} document class is based on the \texttt{article} class.
The main text font is set to IBM Plex Sans and the math font is to IBM Plex Sans
where it is appropriate.
The first page of the template is \textit{Instructions for Policy Hub},
which is suppressed in final document.
The second page is \textit{metadata} information provided
by author(s). The page is suppressed in final document as well.

\section{Options}

\begin{description}
\item[final] - do not print the \textit{Instructions for Policy Hub} page and
  the \textit{metadata} page.
\item[draft] - the opposite of ``final'' option. Prints \textit{Instructions for Policy Hub} page
  and \textit{metadata} page. \textit{Enabled by default.}
\end{description}

\section{Instructions for Policy Hub}

\textit{Instructions for Policy Hub} page contains questions
that author(s) are kindly asked to answer.
The page is surpressed in final document.

\section{Metadata}

Metadata consists of paper \textit{title}, \textit{subtitle} if available,  \textit{authors}
with their \textit{affilitions}, \textit{abstract} or summary,
\textit{key findings}, \textit{center affiliation}, \textit{keywords} and \textit{doi number}.


\DescribeMacro{\title}%
Set title of the paper  in |\title{}| command before |\begin{document}|.

\DescribeMacro{\subtitle}
Set subtitle (if any) within |\subtitle{}| command.

\DescribeEnv{abstract}
Use  |\begin{abstract}...\end{abstract}| environment
instead of  the |\abstract{}| macro to set an abstract or a summary.

\DescribeMacro{\author}
Name(s), affiliations, and email(s) of the author should set in appropriate
macros within |author| command.
The following coding should be applied for each author separately:
\begin{Verbatim}
\author{%
  \person{Firstname Surname},
  \affiliation{Affiliation}
  \email{surname1@email.com}% not printed
}
\end{Verbatim}

Use comma as a separator for multiple affiliations or email addresses.

\DescribeEnv{findings}
The |findings| environment is defined as an unnumbered list,
where findings is set as list items.

\DescribeMacro{\affil}
This command is meant for Center Affiliation.

\DescribeMacro{\keywords}
 The command
 \cs{keywords}\marg{keyword1, keyword2,\ldots} sets keywords for the
 paper. Use comma as a keyword separator. The command \cs{keywords} can have  one optional argument which you can use,  e.g, for  JEL classification codes:
 \begin{Verbatim}
   \keywords[JEL Classification]{Code1, Code2, ...}
\end{Verbatim}

\DescribeMacro{\doi}
\DescribeMacro{\articlenumber}
The |doi| and the paper |number| should be set in \cs{doi} and \cs{articlenumber}
commands respectively before \cs{maketitle} command.

\DescribeEnv{aboutauthor}
The authors' biographies should be set in the |aboutauthors| environment.
The command |\person| will set the name in bold font.

\DescribeEnv{acks}

Acknowledgments should be set in |acks| environment.

\section{Mathematics}

The template loads |amsmath| \cite{amsmath} by default.
You are encourage to use the features provided by this package.

\section{Tables}

Avoid using vertical rules for tables as per the |booktabs| \cite{booktabs} package recomendations.
The package is loaded by default.
Table headers are set with |\thead| command, which gives a slightly bigger font size.
Table notes can be put after |tabular| environment using \cs{note} command. Optional argument to
\cs{note} will change the default preceeding text (\textit{Note:}).
Basic table structure is as follows (\textit{see sample.tex file for more elaborated examples}):
\definecolor{verbc}{gray}{.6}
\begin{Verbatim}[commandchars=+\[\]]
  \begin{table}
    \caption{...}\label{...}
    \begin{tabular}{+color[verbc][<column-specs>]}
      \hline
      +textbf[\thead]{ headerA } & +textbf[\thead]{ headerB } ... \\
      \hline
             row2 c1   & row2 c2            ... \\
             row3 c1   & row3 c2            ... \\
             .......   & .......            ....\\
      \hline
    \end{tabular}
  \end{table}
\end{Verbatim}

\section{Figures}

Figure caption should be before the figure body, therefore \cs{caption} command should preceed
\cs{includegraphics} command. Notes to figure should be set in \cs{note}
  after the figure body, i.e., after \cs{includegraphics} command.
\begin{Verbatim}
  \begin{figure}
    \caption{...}
    \includegraphics{...}
    \note[Source]{...}
  \end{figure}
\end{Verbatim}


\section{Theorems and alike}
You can use either |amsthm| \cite{amsthm} package or |ntheorem| \cite{ntheorem} package for setting theorems. 

\section{Citing references}%
\label{sec:citations}

The template uses |natbib| \cite{natbib} package to format references. 
The citations are set in the author-year format using
\textit{American Economic Association} citation style.
For bibtex users the bibtex style file (\texttt{aea-edt.bst}) is supplied with this bundle:
\begin{Verbatim}[commandchars=+\[\]]
  \bibliographystyle{aea-edt} [+color[verbc]% load bibstyle aea-edt.bst ]
  \bibliography{bibdata}      [+color[verbc]% load +textbf[your] bibdata.bib database file]
\end{Verbatim}

How to use |bibtex| one can refer to \url{http://www.bibtex.org/Using/} or
\url{https://latex-tutorial.com/tutorials/bibtex/}.


\begin{thebibliography}{00}
\bibitem{amsmath} AMS, LaTeX project, \textit{User’s Guide for the amsmath Package}, v. 2.1,
   \url{https://ctan.org/pkg/amsmath}.

\bibitem{booktabs} Simon Fear, \textit{Publication quality tables in LATEX}, January 2020,
  \url{https://ctan.org/pkg/booktabs}.

\bibitem{amsthm} AMS, \textit{Using the amsthm Package}, v. 2.20.3,  \url{https://ctan.org/pkg/amsthm}

\bibitem{ntheorem} Wolfgang May, Andreas Schedler, \textit{An Extension of the LATEX-Theorem Evironment}, v. 1.33, \url{https://ctan.org/pkg/ntheorem}

\bibitem{natbib} Patrick W. Daly, \textit{Natural Sciences Citations and References}, 2010,
  \url{https://ctan.org/pkg/natbib}
\end{thebibliography}
\end{document}

