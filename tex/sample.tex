% Options: [nometapage,noinstructionpage,final]
\documentclass{frbaart}
\usepackage[leqno]{amsmath}
\usepackage{natbib}
\usepackage{lipsum}
\usepackage{blindtext}
\usepackage{booktabs}
\usepackage{tabularx}
\usepackage{float}
\usepackage[table]{xcolor}
\usepackage{colortbl}
\usepackage{graphicx}

\usepackage[T1]{fontenc}
\usepackage{multicol}
\usepackage{marvosym}


%
% metadata
%
\articlenumber{0000-0}

\doi{https://doi.org/10.29338/xx0000-0}


\title{Paper Title}
%\subtitle{Paper Subtitle}


%
% local definition
%
\def\arraystretch{1.25}


\begin{document}


\begin{instructionspage}

  \item
    For the Webpage, please select the topics that you would like your paper to be listed under
    (choose all that apply).
    If there is a topic you would like your paper to fall under but do not see it here,
    please send it to the member of the research editorial team member you are working with.
  
  % To select a  topics, please uncomment \selectnext macro before an item
  \begin{multicols}{2}

%    \selectnext
    \checkbox Banking

%    \selectnext
    \checkbox Capital and Investment

%    \selectnext
    \checkbox COVID

%    \selectnext
    \checkbox Economic growth and development

%    \selectnext
    \checkbox Economic indicators

    \selectnext
    \checkbox Education

%    \selectnext
    \checkbox Exchange rates

%    \selectnext
    \checkbox Financial markets

%    \selectnext
    \checkbox Fiscal policy

%    \selectnext
    \checkbox Forecasting

%    \selectnext
    \checkbox Government programs

%    \selectnext
    \checkbox Health

%    \selectnext
    \checkbox Housing affordability

%    \selectnext
    \checkbox Immigration

%    \selectnext
    \checkbox International trade

%    \selectnext
    \checkbox Labor markets

%    \selectnext
    \checkbox Latin America

%    \selectnext
    \checkbox Model risk

%    \selectnext
    \checkbox Monetary policy

%    \selectnext
    \checkbox Payments

%    \selectnext
    \checkbox Productivity

%    \selectnext
    \checkbox Racial equity

%    \selectnext
    \checkbox Real estate

%    \selectnext
    \checkbox Regulation

%    \selectnext
    \checkbox Small business

%    \selectnext
    \checkbox Technology

%    \selectnext
    \checkbox Uncertainty

%    \selectnext
    \checkbox Wages and compensation

    \selectnext
    \checkbox My own topic

%    \selectnext
%    \checkbox <Your own topic 1>

%    \selectnext
%    \checkbox <Your own topic 2>
    
    
    \end{multicols}

  \item
    \underline{Social Media}:
    Please share the idea or finding from your paper that you think will be most interesting
    to the Bank’s social media follower, something that would be very interesting
    to a well-informed, but general audience.
    We would try to use this in the posts for your paper.

    \begin{response}

     Your response \ldots
     
    \end{response}

    
  \item
    Are there any people in the \underline{media} that you know are writing
    about the topics that your paper addresses that should
    be contacted regarding your paper?

    \begin{response}

     Your response \ldots
      
    \end{response}

    
  \item
    The rest of the information needed for your
    paper is in the following pages.
    Please use the format laid out in the template.
    
\end{instructionspage}



\begin{abstract}
  Lorem ipsum dolor sit amet, consectetur adipiscing
  elit. Pellentesque ut leo elementum, malesuada sem quis, mollis
  neque. In fermentum nunc eget nunc gravida, quis condimentum risus
  blandit. Cras suscipit cursus faucibus. Maecenas ex libero, pretium
  eget libero nec, eleifend fringilla dolor. Vivamus auctor metus
  elit. Nunc vestibulum orci sit amet lorem bibendum, non maximus quam
  porta. Fusce non elit elit. Vivamus porta massa quis velit aliquam,
  in eleifend sem vehicula. Etiam commodo mi semper lacus ultricies,
  sed mollis sem consectetur. Nunc sit amet nibh ac ante suscipit
  mattis. Proin sodales lorem eu molestie tincidunt. Pellentesque
  habitant morbi tristique senectus et netus et malesuada fames ac
  turpis egestas. In vehicula et nunc non auctor. Ut imperdiet
  lobortis dolor nec tincidunt. Mauris tristique egestas augue a
  bibendum. Aenean nec consectetur orci. Duis auctor elit ut
  sollicitudin pretium. Donec nulla mauris, pellentesque in placerat
  in, ullamcorper quis dolor. In vitae rhoncus ligula. Integer.
\end{abstract}

\author{
  \person{Firstname Surname}, % full name of the author
  \affiliation{Affiliation} 
  \email{surname1@email.com} % not printed
}

\author{
  \person{Firstname Surname}, % full name of the author
  \affiliation{Affiliation}
  \email{surname2@email.com} % not printed
}

%
% key findings
%
\begin{findings}
\item
  \lipsum[1][1]
\item
  \lipsum[1][2]
  
\end{findings}

\affil{\lipsum[1][3]}

%\affil{Center for Quantitative Economic Research}
\keywords[JEL Classification]{<Code1>, <Code2>, \ldots}
\keywords{%
  keyword1, keyword2, keyword3, \ldots
}

\maketitle

\begin{aboutauthors}
  \person{Firstname Surname}
  Insert Bio Sketch.

  \person{Firstname Surname}
  Insert Bio Sketch.
\end{aboutauthors}
%
% Acknowledgments:
%
\begin{acks}
  The author would like to thank  insert text.
  The views expressed here are the author’s and not necessarily those of the
  Federal Reserve Bank of Atlanta or the Federal Reserve System.
  Any remaining errors are the author's responsibility.
\end{acks}

%
% TODO acks is not printed with TL2021
%
%\printacks

\textit{Comments to the authors are welcome at
  \textup{\email{surname1@email.com}} or
  \textup{\email{surname2@email.com}}}.


\mainmatter

\section{Section title}

\lipsum[1][1-3]

\subsection{Subsection title}
\lipsum[1]
\footnote{\lipsum[2] }
\lipsum[1][1]\footnote{\lipsum[3][1]}
\lipsum[1-3]

\section{Section title}

\subsection{Subsection title}

\lipsum[1][1-2]

\subsubsection{Subsubsection title}

\lipsum[2][1-2]

\paragraph{Paragraph title}

\lipsum[3][1-2]

\subparagraph{Subparagraph title}

\lipsum[4][1]\footnote{\lipsum[1] }
\lipsum[5][1]\footnote{\lipsum[6][1] }
\lipsum[1-3]


\section{Math}%
\label{sec:math}

The \texttt{amsmath} package is used to set displayed mathematics

\begin{equation*}
  x^n + y^n = z^n
\end{equation*}

\begin{equation}
  \label{eq:cdf}
  \Pr(Z \le t) = \Phi\left(\frac{Z - \mu}{\sigma} \right),
\end{equation}
where $Z$ follows a $N(\mu, \sigma^2)$ distribution.
Equations can be referenced by \texttt{$\backslash$ref}.
When $\mu = 0$ and $\sigma = 1$, the $Z$ in equation~\ref{eq:cdf}
becomes a standard normal variable


Multiline equations can be presented with the \texttt{align}
environment. For example,
\begin{align}
  g_{\mu}(\phi) = 0,\\
  g_{\mu}(X) = 1.
\end{align}


An equation that is not referenced should not be labeled. The starred
version of the \texttt{equation} and \texttt{align} are for this purpose.


\section{Lists: enumerate, itemize, desciption}

\begin{enumerate}
\item first item.
  \lipsum[3][1-4]
\item second item
\item third item
\end{enumerate}

\begin{itemize}
\item first
    \lipsum[3][1-4]
\item second item
\item third item
\end{itemize}


\begin{description}
\item[first] item
\item[second] item
\item[third]
  \lipsum[3][1-4]
\end{description}

\subsection{Subsection}

\lipsum[1][1-3]

\begin{enumerate}
  \item first    \lipsum[3][1-4]
  \item second item
  \item third item
\end{enumerate}

\lipsum[1][1-3]
  
\section{Tables}%
\label{sec:tabs}

\begin{table}[H]
  \caption{Short title}
    \label{tab:realdata}
    %\centering
    \begin{tabular*}{26pc}{llrrrr}
% row 1
    \hline
      \multicolumn{1}{l}{\thead{ Tempus }}
    & \multicolumn{1}{l}{\thead{ Parametri }}
    & \multicolumn{2}{c}{\thead{ Duo-pars modum }}
    & \multicolumn{2}{c}{\thead{ Modum marginalis }} \\
    \hline
% row 2
      &
      & EST
      & SE
      & EST
      & SE
      \\
% row 3
      Aestas
      & $\lambda_1$
      & 2.841
      & 0.459
      & 1.090
      & 0.280
      \\
      & $\lambda_0$
      & 0.179
      & 0.014
      & 0.158
      & 0.015
      \\
      & $\sigma$
      & 1.335
      & 0.106
      & 0.999
      & 0.104
      \\
      & $\sigma_\epsilon (\times 10^{-2})$
      & 1.854
      & 0.087
      & 1.879
      & 0.078
      \\
      Hiems
      & $\lambda_1$
      & 6.225
      & 0.825
      & 4.720
      & 0.711
      \\
      & $\lambda_0$
      & 0.118
      & 0.010
      & 0.114
      & 0.009
      \\
      & $\sigma$
      & 1.506
      & 0.095
      & 1.454
      & 0.089
      \\
      & $\sigma_\epsilon (\times 10^{-2})$
      & 0.908
      & 0.036
      & 0.934
      & 0.043 \\
    \hline
    \end{tabular*}
    \par
    \note{\lipsum[9][1] }
    \note[Source]{\lipsum[9][2] }
\end{table}


\begin{table}[H]
  \caption{ Long title. \lipsum[1][1-4]}%
    \begin{tabularx}{\textwidth}{XXrr>{\extracolsep{1pc}}rr}
% row1
    \hline
      \multicolumn{1}{l}{\thead{ Tempus }}
    & \multicolumn{1}{l}{\thead{ Parametri }}
    & \multicolumn{2}{c}{\thead{ Duo-pars modum }}
    & \multicolumn{2}{c}{\thead{ Modum marginalis }} \\
    \hline
% row 2
      &
      & EST
      & SE
      & EST
      & SE
      \\
% row3
      Aestas
      & $\lambda_1$
      & 2.841
      & 0.459
      & 1.090
      & 0.280
      \\
      & $\lambda_0$
      & 0.179
      & 0.014
      & 0.158
      & 0.015
      \\
      & $\sigma$
      & 1.335
      & 0.106
      & 0.999
      & 0.104
      \\
      & $\sigma_\epsilon (\times 10^{-2})$
      & 1.854
      & 0.087
      & 1.879
      & 0.078
      \\
      Hiems
      & $\lambda_1$
      & 6.225
      & 0.825
      & 4.720
      & 0.711
      \\
      & $\lambda_0$
      & 0.118
      & 0.010
      & 0.114
      & 0.009
      \\
      & $\sigma$
      & 1.506
      & 0.095
      & 1.454
      & 0.089
      \\
      & $\sigma_\epsilon (\times 10^{-2})$
      & 0.908
      & 0.036
      & 0.934
      & 0.043
      \\
      \hline
  \end{tabularx}
  \note{\lipsum[9][1] }
  \note[Source]{\lipsum[9][2] }
\end{table}

\section{Figures}%
\label{sec:figs}

Vector graphics do not lose clarity when being scaled. Make your
figure in pdf format when you first generate it and keep in mind its
sizes in the article to avoid over-scaling. Do not simply convert a
jpeg or png image to a pdf.
\begin{figure}[H]
  \caption{Short title}
  \includegraphics{dice}
  \note{\lipsum[9][1] }
  \note[Source]{\lipsum[9][2] }
\end{figure}

\begin{figure}[H]
  \caption{ Long title. \lipsum[1][1-4]}%
  \includegraphics[width=\textwidth]{dice}
  \note{\lipsum[9][1] }
  \note[Source]{\lipsum[9][2] }
\end{figure}

\section{Citing references}

\cite{carlin1992monte}
\cite{carlin1992monte,gamado2017estimation}
\nocite{*}

\bibliographystyle{aea-edt}
\bibliography{biblio}

\end{document}
 

